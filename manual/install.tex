\chapter{Installation}
\noindent
This chapter lists the product requirements and the installation
notes for SDR.

% %%%%%%%%%%%%
% REQUIREMENTS
% %%%%%%%%%%%%
\section{Requirements}
\noindent
Before running SDR, make sure all product requirements
are satisfied by checking all your servers, where
you plan to install the software. SDR requires to be installed 
on each server you plan to collect data from and one additional 
server, where you plan to keep all recorded data, the reporting server.

\subsection*{Recording Module}
This modules contains the Perl5 environment, OpenSSL and some
additional libraries and all SDR recorders and utilities.

In addition, some Perl modules specific to certain operating systems are as well 
found under recording module, for example: Sun::Solaris::Kstat and
Linux::System::Statistics.

\begin{center}
\begin{tabular}{lll}

\multicolumn{3}{c}{} \\
 \rowcolor{blue!25}
 \textbf{Item} & \textbf{Minimum Requirement} & \textbf{Description} \\
 \hline
 \newline

\multirow{1}{*}{\small{Memory}} &
 \small{128MB RAM} & \small{standard and specialized recorders} \\
 \hline

\multirow{2}{*}{\small{Storage}} &
 \small{150MB} & \small{default installation: 150MB} \\ &
 \small{1MB} & \small{raw data: per day raw data, per host} \\
 \hline

\multirow{4}{*}{\small{Operating Systems}} &
 \small{} & \small{RedHat Entreprise Linux 5.x,6.x} \\ &
 \small{Linux 2.6+} & \small{Ubuntu Server 10.x,11,12+ LTS} \\ &
 \small{} & \small{Ubuntu Server 10.x,11,12+ LTS} \\ &
 \small{Solaris 10} & \small{Solaris 10 Update 6, x64, SPARC} \\
 \hline

\multirow{1}{*}{\small{System Libraries}} &
 \small{zlib} & \small{Linux based operating systems} \\
 \hline

\multirow{2}{*}{\small{Shell}} &
 \small{ksh93} & \small{Linux based operating systems} \\ &
 \small{ksh88} & \small{Solaris} \\
 \hline

\multirow{1}{*}{\small{Utilities}} &
 \small{chkconfig} & \small{Linux based operating systems} \\ &
 \small{cron} & \small{update-rc.d cron defaults} \\
 \hline

\multirow{1}{*}{\small{Java}} &
 \small{Java 6,7} & \small{Linux,Solaris} \\
 \hline

\multirow{1}{*}{\small{Virtualization}} &
 \small{SUNWzoner,SUNWzoneu} & \small{Solaris zones}

\end{tabular}
\end{center}



\subsection*{Reporting}
\noindent
These are the product requierements for SDR Reporting module for all supported
operating systems. RedHat/Ubuntu, Solaris SDR reporting module can be 
installed under RedHat/Ubuntu or Solaris based systems. Currently
the reporting module has been tested under Solaris 10 Update 8 x64 and
SPARC V9. SDR Reporting ships with its own Perl distribution, to avoid conflicts 
with the OS vendor installation.

\begin{center}
\begin{tabular}{lll}

\multicolumn{3}{c}{} \\
 \rowcolor{blue!25}
 \textbf{Item} & \textbf{Minimum Requirement} & \textbf{Description} \\
 \hline
 \newline

\multirow{1}{*}{\small{Memory}} &
 \small{2GB RAM} & \small{} \\
 \hline

\multirow{2}{*}{\small{Storage}} &
 \small{1GB} & \small{default installation} \\ &
 \small{1MB} & \small{raw data: per day raw data, per host} \\
 \hline

\multirow{4}{*}{\small{Operating Systems}} &
 \small{} & \small{RedHat Entreprise Linux 5.x,6.x} \\ &
 \small{Linux 2.6+} & \small{Ubuntu Server 10.x,11,12+ LTS} \\ &
 \small{} & \small{Ubuntu Server 10.x,11,12+ LTS} \\ &
 \small{Solaris 10} & \small{Solaris 10 Update 6, x64, SPARC} \\
 \hline

\multirow{2}{*}{\small{System Libraries}} &
 \small{zlib, expat, bzip2-libs, libX11} & \small{Linux based operating systems} \\ &
 \small{libXt, libICE, libSM, libXau, libXdmcp} & \small{} \\
 \hline

\multirow{2}{*}{\small{Shell}} &
 \small{ksh93} & \small{Linux based operating systems} \\ &
 \small{ksh88} & \small{Solaris} \\
 \hline

\multirow{1}{*}{\small{Utilities}} &
 \small{chkconfig} & \small{Linux based operating systems} \\ &
 \small{cron} & \small{update-rc.d cron defaults} \\
 \hline

\end{tabular}
\end{center}


% %%%%%%%%%%%% %
% INSTALLATION %
% %%%%%%%%%%%% %
\pagebreak
\section{Installing SDR}
\noindent
In order to operate with SDR you will need to install 
the recording package on all your servers you plan to 
collect data from. You need as well to configure how 
each server's data will be stored and the data 
retention policy: how many days you plan to keep 
data on those servers and finally a last step setting up
a reporting server, a place you will store the recorded 
data over time.

\subsection*{Recording}


% LINUX
\subsubsection*{Linux}
This is the main installation procedure, in setting up
the SDR Recording module. Execute the following steps,
on each physical server:

\begin{enumerate}

\item SDR package\\
SDR software should be configured to run as sdr user. Create a specific 
username/group for SDR. Select 'sdr' for a default installation.

\begin{Verbatim}[fontsize=\relsize{-2},frame=single,
                 label=\fbox{Installation Package},
                 framesep=3mm,labelposition=bottomline]

 # groupadd sdr
 # useradd -d /home/sdr -g sdr -m sdr
     
 # cd /opt
 # bzcat sdr.074.linux-x64.tar.bz2 | tar xvf -

\end{Verbatim}


\item Startup Services\\ 
Edit the startup script, sdr. Enable/Disable services for RECORDERS variable.
Simple add or remove under RECORDERS variable the names of the records
you want to start/stop. Ignore the last part 'rec'.

\begin{Verbatim}[fontsize=\relsize{-2},frame=single,
                label=\fbox{Startup Services},
                framesep=2mm,labelposition=bottomline]

Default:
 RECORDERS="sys cpu nic"

Add jvmrec, netrec:
 RECORDERS="sys cpu nic net jvm"

\end{Verbatim}

\item Startup\\
Enable sdr startup script

\begin{Verbatim}[fontsize=\relsize{-2},frame=single,
                label=\fbox{Startup},
                framesep=2mm,labelposition=bottomline]

   # cd /etc/init.d
   # ln -s /opt/sdr/etc/sdr .
   # chkconfig --add sdr
   # chkconfig --list sdr
   
   sdr  0:off  1:off  2:on   3:on   4:on   5:on   6:off  

 Start all configured services

   # /etc/init.d/sdr start
   Starting SDR services
    sysrec service: ok
    cpurec service: ok
    netrec service: ok


 Stop all configured services

   # /etc/init.d/sdr stop
   Stopping SDR services
    sysrec service: ok
    cpurec service: ok
    netrec service: ok

\end{Verbatim}

\item Logrotate

Enable log rotation every 24hrs. Configure sdr rotation script 
and raw2day. Edit root's or sdr's crontab, depending what
is your user configuration for running SDR. Example root:

\begin{Verbatim}[fontsize=\relsize{-2},frame=single,
                label=\fbox{Logrotate},
                framesep=2mm,labelposition=bottomline]

# env EDITOR=vi crontab -e
05 00 * * *  /usr/sbin/logrotate -f -s /opt/sdr/log/logsdr.status /opt/sdr/etc/logrotate.sdr
06 00 * * *  /opt/sdr/bin/raw2day


\end{Verbatim}
\end{enumerate}

SDR uses its own logroate job in order to be flexible and dont conflict with 
other operating system jobs. raw2day as well is called after the logrotation 
has taken place ! If your installation requires to copy every night the raw 
data to the reporting system, make sure you configure raw2day and check the 
SDR Manual for more information.    



% SOLARIS 
\subsubsection*{Solaris}

This is the main installation procedure, in setting up
the SDR Recording module. Execute the following steps,
on each physical server. Example for SDR 0.74 release:

\begin{enumerate}

\item SDR package:

\begin{Verbatim}[fontsize=\relsize{-2},frame=single,
                 label=\fbox{Installation procedure},
                 framesep=3mm,labelposition=bottomline]

$ wget \
  http://www.systemdatarecorder.org/pkgs/sdr-074-solaris10x64.tar.bz2

# cd /opt
# bzcat sdr-074-solaris10x64.tar.bz2 | tar xvf -

\end{Verbatim}

\item setenv script:\\
Adjust SDR\_ROOT to point to your SDR installation
path. Default is /opt/sdr

\item SAR:\\
SDR uses SAR, system activity reporter

\begin{Verbatim}[fontsize=\relsize{-2},frame=single,
                 label=\fbox{SAR},
                 framesep=3mm,labelposition=bottomline]

# svcadm enable sar

# crontab -l sys
0,5,10,15,20,25,30,35,40,45,50,55 * * * * /usr/lib/sa/sa1 

\end{Verbatim}


\noindent
On Solaris systems, one very useful feature of SDR is SMF: 
the service management facility, a way to automatic restart 
your service in case of a failure, maintenance, etc. SDR 
uses SMF, by default, under Solaris 10 but you need to
activate it. Below there are the steps used to enable SDR under SMF.

\item SMF Manifests:

\begin{Verbatim}[fontsize=\relsize{-2},frame=single,
                 label=\fbox{SMF Manifests},
                 framesep=3mm,labelposition=bottomline]

# svcs -a | grep rec

# svcadm disable sysrec
# svcadm disable cpurec
# svcadm disable nicrec

# cd /opt/sdr/smf
# svccfg validate sysrec.xml 
# svccfg validate cpurec.xml                       
# svccfg validate nicrec.xml                       

# svccfg  import sysrec.xml
# svccfg  import cpurec.xml                        
# svccfg  import nicrec.xml                        

# svcadm enable sysrec
# svcadm enable cpurec
# svcadm enable nicrec                             

\end{Verbatim}

\item Check recorders and raw data:

\begin{Verbatim}[fontsize=\relsize{-2},frame=single,
                 label=\fbox{Pre-check},
                 framesep=3mm,labelposition=bottomline]

# svcs -a | grep rec
online         16:36:07 svc:/application/sysrec:default
online         16:37:43 svc:/application/cpurec:default
online         16:37:47 svc:/application/nicrec:default

# pwd
/opt/sdr/log/raw
# ls -lrt
total 25
-rw-r--r--   1 root     root         342 Oct 31 16:39 cpurec.raw
-rw-r--r--   1 root     root         522 Oct 31 16:39 nicrec.raw
-rw-r--r--   1 root     root         263 Oct 31 16:40 sysrec.raw

\end{Verbatim}

\noindent
Enable for each raw file, a entry for logadm to rotate the file
at midnight and to compress the file. For this make sure you are
superuser and modify the /etc/logadm.conf file or use logadm
utility to add the entries.

\item Enable logadm:

\begin{Verbatim}[fontsize=\relsize{-2},frame=single,
                 label=\fbox{Logadm},
                 framesep=3mm,labelposition=bottomline]

# SDR Monitoring
/opt/sdr/log/raw/sysrec.raw -c -p 1d -z 0
/opt/sdr/log/raw/cpurec.raw -c -p 1d -z 0
/opt/sdr/log/raw/nicrec.raw -c -p 1d -z 0

# logadm -V 
\end{Verbatim}


\item root crontab:

\begin{Verbatim}[fontsize=\relsize{-2},frame=single,
                 label=\fbox{crontab},
                 framesep=3mm,labelposition=bottomline]

# crontab -e

Add here logadm to be done at 00:05, everyday instead of 3AM
and move the raw data compressed into daily directories.

#
05 00 * * * /usr/sbin/logadm
10 00 * * * /opt/sdr/bin/raw2day

\end{Verbatim}

\end{enumerate}




% %%%%%%% %
% UPGRADE %
% %%%%%%% %

\pagebreak
\section{Upgrading SDR}
\noindent
This section describes the practical steps to keep up to
date SDR.

\subsection*{Recording}
\noindent
If you have already installed SDR make sure you follow 
the instructions.


% LINUX
\subsubsection*{Linux}

\begin{enumerate}

\item Backup config and hook files:
\begin{Verbatim}[fontsize=\relsize{-2},frame=single,
                 label=\fbox{SDR Upgrading procedure},
                 framesep=3mm,labelposition=bottomline]

# cd /opt/sdr
# cp -pr bin bin.sdr.previous

\end{Verbatim}


\item Disable each service:
\begin{Verbatim}[fontsize=\relsize{-2},frame=single,
                 label=\fbox{Disable Services},
                 framesep=3mm,labelposition=bottomline]

# /etc/init.d/sdr stop 

\end{Verbatim}


\item SDR package:

\begin{Verbatim}[fontsize=\relsize{-2},frame=single,
                 label=\fbox{Installation Package},
                 framesep=3mm,labelposition=bottomline]

$ wget \
  http://www.systemdatarecorder.org/pkgs/sdr-0734-linux-x64.tar.bz2

# cd /opt
# bzcat sdr-0734-linux-x64.tar.bz2 | tar xvf -

\end{Verbatim}

\item setenv script
\noindent
Restore MONITOR\_PATH to point to your SDR installation
path. Default is /opt/sdr

\item Restore ftp, ssh2 hooks
\noindent
raw2day is used to automatically send data every night 
to a reporting server. If you use a 24hours time window 
update policy make sure you restore your raw2day hooks 
as they were found before the upgrade.

\end{enumerate}



% SOLARIS 
\subsubsection*{Solaris}

\begin{enumerate}

\item Backup config and hook files:
\begin{Verbatim}[fontsize=\relsize{-2},frame=single,
                 label=\fbox{SDR Upgrading procedure},
                 framesep=3mm,labelposition=bottomline]

# cd /opt/sdr
# cp -pr bin bin.sdr.previous

\end{Verbatim}


\item Disable each service:
\begin{Verbatim}[fontsize=\relsize{-2},frame=single,
                 label=\fbox{Disable Services},
                 framesep=3mm,labelposition=bottomline]

# svcadm disable sysrec
# svcadm disable cpurec
# svcadm disable nicrec

\end{Verbatim}


\item SDR package:

\begin{Verbatim}[fontsize=\relsize{-2},frame=single,
                 label=\fbox{Installation Package},
                 framesep=3mm,labelposition=bottomline]

$ wget \
  http://www.systemdatarecorder.org/pkgs/sdr-0734-solaris10x64.tar.bz2

# cd /opt
# bzcat sdr-0734-solaris10x64.tar.bz2 | tar xvf -

\end{Verbatim}

\item setenv script
\noindent
Restore MONITOR\_PATH to point to your SDR installation
path. Default is /opt/sdr

\item Restore ftp, ssh2 hooks
\noindent
raw2day is used to automatically send data every night 
to a reporting server. If you use a 24hours time window 
update policy make sure you restore your raw2day hooks 
as they were found before the upgrade.

\end{enumerate}






% %%%%%%%%% %
% UNINSTALL %
% %%%%%%%%% %
\pagebreak
\section{Uninstalling SDR}

\subsection*{Recording}
\noindent
This section describes how to remove SDR from a Linux or
Solaris system.

% LINUX
\subsubsection*{Linux}

\begin{enumerate}

\item Stop all recorders:
\begin{Verbatim}[fontsize=\relsize{-2},frame=single,
                 label=\fbox{Stop all recorders},
                 framesep=3mm,labelposition=bottomline]

# /etc/init.d/sdr stop

\end{Verbatim}


\item Remove all SDR software:
\begin{Verbatim}[fontsize=\relsize{-2},frame=single,
                 label=\fbox{Removal of SDR},
                 framesep=3mm,labelposition=bottomline]


# cd /opt/
# rm -rf sdr

\end{Verbatim}
\end{enumerate}



% Solaris 
\subsubsection*{Solaris}
In order to remove SDR from your system make sure you login in the 
global zone of the physical machine where you have installed the software. 
Become root and disable all SDR recorders from SMF, if you are on Solaris 
and delete all its manifests.  Make sure you delete the log directory, 
where SDR keeps its recorded data to make room for other applications.


\begin{enumerate}

\item Stop all recorders:
\begin{Verbatim}[fontsize=\relsize{-2},frame=single,
                 label=\fbox{SDR Uninstall procedure},
                 framesep=3mm,labelposition=bottomline]

# svcadm disable sysrec
# svcadm disable cpurec                            
# svcadm disable nicrec                            

# svcs -a | grep rec
disabled       16:43:55 svc:/application/sysrec:default
disabled       16:44:02 svc:/application/cpurec:default
disabled       16:44:05 svc:/application/nicrec:default

\end{Verbatim}


\item Delete all SDR manifests:
\begin{Verbatim}[fontsize=\relsize{-2},frame=single,
                 label=\fbox{Removal SMF manifests},
                 framesep=3mm,labelposition=bottomline]

# svccfg delete application/sysrec
# svccfg delete application/cpurec                 
# svccfg delete application/nicrec                 

At this moment SMF does not know anymore about SDR

# svcs -a | grep rec

# cd /opt/
# rm -rf sdr

\end{Verbatim}
\end{enumerate}


\endinput
